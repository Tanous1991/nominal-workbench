\begin{frame}{Toplevel}

Le logiciel est accompagné d'un mode interactif permettant à
l'utilisateur de manipuler directement le système en rajoutant règles,
stratégies ou encore la réduction directe de terme.

\end{frame}

\begin{frame}{Directives}

Plusieurs directives sont à la disposition de l'utilisateur pour lui
permettre d'affiner le développement de son système de réécriture.
\bigskip

On peut ainsi : 
\begin{itemize}
\item Effectuer des tests sur une expression selon plusieurs modalités
\item Afficher le type d'un terme
\item Exporter la représentation d'un terme \emph{hash-consé} vers un graphe
\end{itemize}

\end{frame}

\begin{frame}[fragile]{Tests}

Nous disposons de plusieurs outils pour tester un programme.
\medskip

\begin{verbatim} 
:test <terme> <predicat> <resultat> 
\end{verbatim}

Par exemple, sur un système décrivant les axiomes de Peano. Pour
tester que l'addition se comporte correctement nous pourrons écrire :

\begin{verbatim}
:test rewrite Add(Successor(Zero), Successor(Zero)) 
      with BottomupAll --equal Successor(Successor(Zero))
\end{verbatim}

\end{frame}

\begin{frame}[fragile]{Tests}

Nous pouvons également tester l'échec d'un test par une exception pré-définie :
\medskip

Ainsi, le test suivant échouera comme prévu
\begin{verbatim}
:test Successor(Zero, Zero) 
      --failwith TermSystemError.WrongTermArity
\end{verbatim}

\end{frame}

\begin{frame}[fragile]{Tests}
  On peut aussi tester la correspondance d'un terme à un type ou à un
  patron

  \bigskip
  Dans le lambda-calcul, on peut ainsi tester l'alpha-conversion
\begin{verbatim}
:match App(Lambda(x, Var(x)), Lambda(y, Var(y))) 
       --with App(?T, ?T) 
\end{verbatim}

\end{frame}

\begin{frame}{Autres commandes}
  
  On peut également citer les directives utilitaires :
  \bigskip

  \begin{itemize}
  \item :type <t> -- Permet d'afficher le type du terme t
  \bigskip

  \item :dot <t> <fichier> -- Crée un graphe ``dot'' de la version \emph{hash-consé} du term t
  \end{itemize}

\end{frame}
