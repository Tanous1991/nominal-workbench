\begin{frame}{Toplevel}

\begin{block}{Mode interactif}
Permet à l'utilisateur de manipuler directement le système en rajoutant règles,
stratégies ou encore la réduction directe de terme.
\end{block}

\begin{block}{Directives}
\begin{itemize}
\item Effectuer des tests sur une expression selon plusieurs modalités
\item Afficher le type d'un terme
\item Exporter la représentation d'un terme \emph{hash-consé} vers un graphe
\end{itemize}
\end{block}

\end{frame}

\begin{frame}[fragile]{Tests}

\begin{block}{Tester les termes}
\begin{verbatim} 
:test <terme> <predicat> <resultat> 
\end{verbatim}
\end{block}

Tester l'addition dans le système de réécriture des axiomes de Peano :
\begin{verbatim}
:test rewrite Add(Successor(Zero), Successor(Zero)) 
      with BottomupAll --equal Successor(Successor(Zero))
\end{verbatim}

\end{frame}

\begin{frame}[fragile]{Test d'une erreur}

Nous pouvons également tester l'échec d'un test par une erreur pré-définie :
\medskip

\begin{verbatim}
:test Successor(Zero, Zero) 
      --failwith TermSystemError.WrongTermArity
\end{verbatim}

\end{frame}

\begin{frame}[fragile]{Test alpha-conversion}
  On peut aussi tester la correspondance d'un terme à un type ou à un
  pattern.

\begin{block}{Alpha-conversion dans le lambda-calcul}
\begin{verbatim}
:match App(Lambda(x, Var(x)), Lambda(y, Var(y))) 
       --with App(?T, ?T) 
\end{verbatim}
\end{block}

\end{frame}

\begin{frame}{Directives utilitaires}

  \begin{itemize}
  \item \texttt{:type <t>} -- Permet d'afficher le type du terme t
  \bigskip

  \item \texttt{:dot <t> <fichier>} -- Crée un graphe ``dot'' de la version \emph{hash-consé} du term t
  \end{itemize}

\end{frame}
