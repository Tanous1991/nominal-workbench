\begin{frame}
\frametitle{Système de réécriture}

\begin{block}{Étude des systèmes de réécriture}
\begin{itemize}
\item Langage de réécriture et typage ;
\item Algorithme de pattern matching non linéaire ;
\item Hash-consing ;
\item Multiple représentation d'une même donnée ;
\item Langage de stratégie.
\end{itemize}
\end{block}

\end{frame}

\begin{frame}
\frametitle{Travail d'équipe}

\begin{itemize}
\item Répartition des tâches via rallydev ;
\item Utilisation d'un dépôt git.
\end{itemize}

\begin{block}{Réunion développeurs}
\begin{itemize}
\item Environ 1 fois par semaine, bonne fréquence ;
\item Trop rigide par rapport au modèle Scrum au départ ;
\item Réunion et développement dans la même après-midi par la suite.
\end{itemize}
\end{block}
\end{frame}

\begin{frame}
\frametitle{Limites de Scrum}

\begin{block}{Environnement}
\begin{itemize}
\item Peu adapté à du développement sporadique ;
\item Diversité des périodes de pause (examens) ;
\item Diversité des emplois du temps ;
\item Malgré tout, proximité entre les membres de l'équipe.
\end{itemize}
\end{block}
\pause

\end{frame}

\begin{frame}
\frametitle{Limites de Scrum}

\begin{block}{Sujet}
\begin{itemize}
\item Partie recherche peu représentable via la méthode Scrum ;
\item Difficulté d'estimation ;
\item Néanmoins, fortement lié à l'implémentation dans notre cas.
\end{itemize}
\end{block}

\begin{block}{Réunion client}
\begin{itemize}
\item Peu de réunion réalisée mais productive ;
\item Interaction plus fréquente par mail (1 à 1).
\end{itemize}
\end{block}

\end{frame}
