\begin{frame}{Description d'un système de réécriture}
  \begin{block}{Description}
    La description d'un système se fait par des déclarations de 4
    sortes différentes :
    \begin{itemize}
    \item \verb?kind?
    \item \verb?constant?
    \item \verb?operator?
    \item \verb?rule?
    \end{itemize}
  \end{block}
\end{frame}

\begin{frame}[fragile]{Kind}

  Le mot-clé \verb?kind? pour décrire les types manipulés par le
  système de réécriture.
\begin{verbatim}
  kind Integer : type
  kind List : type -> type
  kind Couple : type -> type -> type
  kind Var : atom
\end{verbatim}
\end{frame}

\begin{frame}[fragile]{Constant}
  \begin{itemize}
  \item élément de base du langage
  \item typé avec les types du langage décrit (\verb?kind?)
  \end{itemize}

\begin{verbatim}
  kind Integer : type
  kind Bool : type
  kind List : type -> type
  kind Couple : type -> type -> type

  #generics
  constant Nil : forall(A).List<A>
  constant NilAssoc : forall(A,B).List<Couple<A,B>>

  #instatiated
  constant IntNil : List<Integer>
\end{verbatim}
\end{frame}

\begin{frame}[fragile]{Operator}
  Déclarations similaires aux déclarations de constantes.
\begin{verbatim}
  operator Cons : forall(A).A * List<A> -> List<A>
  operator Hd : forall(A).List<A> -> A
  operator Tl : forall(A).List<A> -> List<A>

  # Hd(Tl(Cons(False, Cons(True, Nil))))
\end{verbatim}
\end{frame}

\begin{frame}[fragile]{Rule}
  \begin{itemize}
  \item L'ordre d'application des règles est définit par les stratégies
  \item Généricité structurelle (\emph{placeholders})
  \end{itemize}

  \begin{verbatim}
    rule [hd]:
      Hd(Cons(?x, ?y)) => ?x

    rule [tl]:
      Tl(Cons(?x, ?y)) => ?y
  \end{verbatim}
\end{frame}
