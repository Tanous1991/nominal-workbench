The hashconsing allows to express the structural equivalence (without
congruence) by only allocating once two terms that are identical. To do so, we
have to completely drop the names of binders and binded variables. The result is
a term that looks a lot like the typed terms, with some subtlety with the
binders and the binded variables.

The type is the following:
%% Lstlisting one
\begin{verbatim}
type 'a hashed = private { id: id; hash : int; value : 'a }

type 'a hlist = 'a hashed list

type hterm_raw = private
  | HConst of ident
  | HTerm of ident * hterm_raw hlist
  | HBinder of id hlist (* refers to the id of the head of an hashconsed
                           list. *)
  | HVar of int
  | HFreeVar of ident
\end{verbatim}

The most difficult part lies in the HBinder and HVar parts. To explain how it is
constructed, the best idea is to explain the algorithm. 

We cross our typed term from left to right, than at the end of the traversal, we
go back to effectively create our hashconsed subterms. The second part is
mandatory: when we create the subterms of a HTerm, the lsit has to be hashconsed
and we have to begin by the tail. Each cell of this hashconsed list is a record
containing a value and an id, which is the id of the actual hashconsed list
where this cell is the head of the list.

Now for the first part, it is moslty to capture the binders. And then while we
go from right to left, we capture the id of its binded variables that are in
hashconsed lists. This way, a binder is represented by an id list, which is
morally its binded variables. We just need to hashcons this id list and the
Binder can be hashconsed.

However, there is still one problem with this method, and this is solved by the
\texttt{HVar of int}. This is basically a de Bruijn index, which is mandatory
unfortunately. For example, imagine the following term 
\texttt{Lambda(x, Lambda(y, App(App(Var(y), Var(y)), App(Var(x), Var(y)))))}.
Without using de Bruijn, by using for example just HVar (that would be
hashconsed once), the following pattern would match: 
\texttt{Lambda(?x, Lambda(?y, App(?T, ?T)))} because this would extract the vars
out of their context and they would became identical. Using the de Bruijn
indices solves this issue.

The hashing functions computes a hash value for each term and list. For list, it
computes from the head and tail. If its an empty list its value is zero, else it
is $head + 17 \times tail$ where $head$ and $tail$ are values given in
argument of the \texttt{hash\_value\_hlist} function. An extract from the code:
\begin{verbatim}
  let rec hash_hlist = function
    | [] -> 0
    | v :: [] -> hash_term v.value
    | v1 :: v2 :: _ -> hash_value_hlist (hash_term v1.value) v2.hash
\end{verbatim}
If you want to modify the hashing function, please change
\texttt{hash\_value\_hlist}, since it is used multiple times.

Finally, you can notice that a hterm is a record with the hashconsed term and a
name field. This is simply a tree mapping the names that were lost during the
hashconsing. This allows to recreate the typed term corresponding to the
hashconsed term.
