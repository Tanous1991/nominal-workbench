% Distributed under the MIT License.
% (See accompanying file LICENSE.txt)
% (C) Copyright Vincent Botbol
% (C) Copyright Yohan Bismuth
% (C) Copyright Pierre Talbot <ptalbot@hyc.io>

\documentclass[12pt,a4paper]{article}

\usepackage[utf8]{inputenc}
\usepackage[american]{babel}
\usepackage[T1]{fontenc}
\usepackage{hyperref}

\title{Nominal Workbench\\
Methodology}
\author{NoWork development team\\[2em]}
\date\today

\begin{document}
\maketitle

\section{Overview}

In this document, we will describe the methodology that we apply for
the development of the project Nominal Workbench. This description
will follow four main axis which are:
\begin{itemize}
\item Agile methodology.
\item The tools used.
\item The interactions with the client.
\item The interactions in the team.
\end{itemize}

\section{Agile methodology}

In this project, we chose to apply an agile methodology for an
efficient development. We had two options in mind, which were Scrum or
Extreme Programming.

We finally decided to choose the Scrum methodology for many reasons:
first of all, the development in Scrum can be done by each developer
independently. In Extreme Programming, dev team has to be in a single
room and sometimes has to work by pair of developers. Moreover,
the interaction with the client is simpler with Scrum. The client has
to talk mainly with the customer relations manager to follow the
evolution of the project while, in Extreme Programming, he has to
interact with every developer to stay aware.

Meetings take place weekly in order to answer those three questions:
\begin{itemize}
\item What have I done ?
\item What is my next work ?
\item What difficulties have been encountered ?
\end{itemize}

We had to adapt the Scrum methodology to match with our schedule and
our disponibility. Therefore, we defined the Sprints duration to be of
one week instead of a month.

\section{Tools}

All along this project, we will use some tools to improve our
efficiency.

\subsection{Rallydev}

\href{http://rally1.rallydev.com/}{Rallydev} is a Scrum platform on which we can add, describe and edit
user stories. We can also assign a developer with a task and see the
progression of the development. A ``story'' can be either defined, in
progress, completed or accepted.

\subsection{Git}

Git is a distributed revision control and source code management system. We decided to use the web-based hosting service: \href{http://github.com/}{GitHub}. This choice was motivated by the fact that both the client and the development team were already familiar with this kind of repository system.

GitHub also provides the possibility to create Wiki pages and to easily report issues thus enhancing the communication between the team members.

The project repository is forked from the client's one and will be asked to be pulled upstream after each ``sprint'' iteration.

\subsection{Building system}

There were several building systems we could use. We decided that
classical Makefile would not suffice as the scalability was difficult
to maintain as the project would grow. After consideration, we finally
decided to use \emph{ocp-build} which allows the project to be
well-organized, to spend less time resolving dependencies issues and
to have an easy way for everyone to include the new tests and
libraries without a deep knowledge about the building system.

\section{Client interactions}

Once every two weeks, or on demand, the client will be able to discuss with some of the team members. A deliverable will be pushed on the git repository of the client at least 3 days before each meeting. During these meetings, a full report of the advancement will be presented to the client. Afterwards, the client reacts if any concern is raised about the work achieved.

If necessary, there will be also a discussion about development
issues. It may be about technological choices or implementation
decisions.

We will conclude the meeting with an overview of the future work that
will be done until the next one. The client will then be able to
request any new features he'd like to add into the project.

If there is an urging matter between two meeting, such as a bug
related issue or an important feature to quickly add then the project
manager will adapt the current sprint and include the new stories with
a suitable priority. Such an event may be signaled either by a mail
or directly on GitHub depending on the client preferences.

\section{Team interactions}

The team members will interact between them with the help of:

\begin{itemize}
\item Weekly Scrum meeting to close the current sprint and open a new one. We'll discuss the difficulties of the previous iteration and our next move.
\item A Google group to discuss specific technical points, so anyone can give its opinion.
\item An IRC channel if a discussion request more interactions.
\item The Wiki is a place to write down tutorials and give pointers to relevant documentation.
\item The GitHub repository will help to communicate about the code but also to centralize the issues, thanks to the bugs tracker.
\item The detailed commit messages in the repository so anyone can read about the last modifications of the project.
\end{itemize}

However, the best interaction we can have is to discuss face-to-face and so the university will be a good place for this.

\end{document}
