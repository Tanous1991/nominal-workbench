\documentclass{article}

\usepackage{verbatim}
\usepackage[utf8]{inputenc}

\title{Nominal Workshop : \\
  Coding style conventions}

\begin{document}

\maketitle

\section{Lexical Conventions}

\subsection{Naming}

\medskip

\begin{itemize}
\item \textbf{Function and variable names}: Lower cases, with underscore to 
  separate the words (e.g. open\_file).
\item \textbf{Object name}: Use the camel case (e.g. FileOpener).
\item \textbf{File names}: The name of any files (e.g. .ml or .mli) must be in
  lower case.
\item \textbf{Module name}: same as functions and variables, with first letter
  in capital. 
\item \textbf{Sum types and exceptions}: should be written as module's names.
\end{itemize}

\subsection{General}

\medskip 

\begin{itemize}
\item \textbf{Indentation}: With two-spaces (don't use tabulation, set your
  editor to use spaces.)
\item \textbf{Line width}: in general, avoid more than 80 characters. It is
  easier to read.
\item \textbf{Tuples}: the comma should always be followed by a space, to ease
  reading. For example, \textsf{(a, b)}, not \textsf{(a,b)}.
\item \textbf{Lists}: the list constructor \textsf{::} is always preceded and
  followed by a space.
\item \textbf{Parenthesis}: avoid unnecessary parenthesis, even for tuples when
  it is not needed. To enclose a imperative-style sequence, prefer \textsf{begin
    .. end}.
\end{itemize}

\section{Indentation conventions}

To avoid indentation problems, simply use a good editor (Tuareg for emacs, for
example). However, here are some conventions to respect:

\subsection*{Pattern-matching}

In the \textsf{match .. with} case, cases should be aligned with the
\textsf{match}:

\begin{verbatim}
let eval ast =
  match ast with
  | Int i -> ...
...
\end{verbatim}

The vertical bar for the first case can be avoided, but the beginning of the
case must be indented once.

In case \textsf{function} is used over \textsf{match .. with}, the cases must be
indented once. For example:

\begin{verbatim}
let eval ast = function
  | Int i -> ...
...
\end{verbatim}

And not:


\begin{verbatim}
let eval ast = function
               | Int i -> ...
...
\end{verbatim}

\subsubsection*{Pattern-matching in a pattern-matching}

\medskip

Use \textsf{begin .. end} around the inner pattern-matching. The use of
parenthesis should be avoided.

\subsection*{The \textsf{let .. in} clause}

The \textsf{in} should be placed at the end of the expression, even in case of
long expressions. For instance:

\begin{verbatim}
...
let r = fun_call
  arg1
  arg2 in
...
\end{verbatim}

Consecutive \textsf{let .. in} share the same indentation:

\begin{verbatim}
...
let a = ... in
let b = ... in
...
\end{verbatim}

And not:

\begin{verbatim}
...
let a = ... in
  let b = ... in
    ...
\end{verbatim}

Actually, any expression after the \textsf{in} should be indented as its
\textsf{let}.

\section{Comments}

Don't comment everything, only tricky parts. Each function in the .mli must be
commented using the ocamldoc format. Here is an example from the standard
library:

\begin{verbatim}
val rev_append : 'a list -> 'a list -> 'a list
(** [List.rev_append l1 l2] reverses [l1] and concatenates it to [l2].
   This is equivalent to {!List.rev}[ l1 @ l2], but [rev_append] is
   tail-recursive and more efficient. *)
\end{verbatim}

Once parsed by the doc generator, the expressions between brackets are
emphasized. This kind of comments shows how to use the function, even if it is
obvious with only the type it can be really useful sometimes. Comments
``à-la-ocamldoc'' are placed after the signature in .mli.

\section{Coding conventions}

\begin{itemize}
\item \textbf{Global variables}: The usage of mutable global variables is
  prohibited.
\item \textbf{Duplication}: There shouldn't be two parts of code doing the same
  things, factorize.
\item \textbf{Interface}: Every .ml file must have the corresponding interface
  file. Its content must be limited to values which are used by other
  modules. 
\item \textbf{Type annotations}: Type inference is really strong in ML, so
  annotations are not useful. They should only be used to help the compiler if
  needed, or for coercion with objects.
\item \textbf{Obj.magic} doesn't exists. If you think you need the module
  \textsf{Obj}, then your solution is wrong.
\end{itemize}

\end{document}
