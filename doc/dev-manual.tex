% Distributed under the MIT License.
% (See accompanying file LICENSE.txt)
% (C) Copyright NoWork team

\documentclass[12pt,a4paper]{article}

\usepackage[utf8]{inputenc}
\usepackage[american]{babel}
\usepackage[T1]{fontenc}

\title{NoWork\\
Developer manual}
\author{NoWork development team\\[2em]}
\date\today

\begin{document}
\maketitle

\section{Introduction}

Briefly present the project and say which preliminaries document the dev should have read first (wiki, user-manual, coding-style,...).

\section{Installing and hack on NoWork}

How to install and begin to develop with NoWork.

\section{Architecture}

Present the directory layout and the different component of the system.

\section{Top-level}

Present the top-level and coding specificities.

\section{Parsing and data representation}

\subsection{System representation}

Which transformation of the representation of the system occurred and why?

\subsection{Term representation}

Which transformation of term occurred and why?

\subsubsection{Hash-consing}
Explain further the hash-consing representation.

\section{Algorithms}

\subsection{Pattern-matching}

\subsection{Rewriting}

\section{Rational}

Important decision that we made such as : Why have we split the language in an interactive and "normal" mode? Why do we have only one parser/lexer (instead of one for system and one for term). It can be technical.


\end{document}
