% Distributed under the MIT License.
% (See accompanying file LICENSE.txt)
% (C) Copyright Pierrick Couderc
% (C) Copyright Mathieu Chailloux
% (C) Copyright Pierre Talbot <ptalbot@hyc.io>

\documentclass[12pt,a4paper]{article}
\usepackage[american]{babel}
\usepackage{verbatim}
\usepackage[utf8]{inputenc}
\usepackage{listings}
\usepackage{color}

\definecolor{purple}{rgb}{0.65, 0.12, 0.82}

\lstnewenvironment{OCaml}
                  {\lstset{
                      language=[Objective]Caml,
                      breaklines=true,
                      commentstyle=\color{purple},
                      stringstyle=\color{red},
                      identifierstyle=\ttfamily,
                      keywordstyle=\color{blue},
                      basicstyle=\footnotesize
                    }
                  }
                  {}

\title{Nominal Workbench \\
  Coding style}
\author{NoWork development team\\[2em]}

\begin{document}

\maketitle

\section{Lexical conventions}

\subsection{Naming}

\medskip

\begin{itemize}
\item \textbf{Function and variable names}: Lower cases, with underscore to 
  separate the words (e.g. open\_file).
\item \textbf{Object name}: Use the CamelCase (e.g. FileOpener).
\item \textbf{File names}: The name of any files (e.g. .ml or .mli) must be in
  lower case, with words separated by underscore.
\item \textbf{Module name}: First letter in upper case, the rest in lower case
  with underscore to separate words.
\item \textbf{Exceptions}: First letter in upper case, the rest in
  lower case with underscore to separate words.
\item \textbf{Sum types}: First letter in upper case, the rest in lower case
  with underscore to separate words.
\end{itemize}

\subsection{General}

\medskip 

\begin{itemize}
\item \textbf{Line width}: In general, avoid more than 80 characters. It is
  easier to read.
\item \textbf{Tuples}: The comma should always be followed by a space, to ease
  reading. For example, \textsf{(a, b)}, not \textsf{(a,b)}.
\item \textbf{Lists and binary operators}: The list constructor \textsf{::} is
  always preceded and followed by a space. Binary operators follow the same
  rule.
\item \textbf{Parenthesis}: Avoid unnecessary parenthesizing, even for tuples when
  it is not needed. To enclose a imperative-style sequence, prefer \textsf{begin
    .. end}.
\end{itemize}

\section{Indentation and syntactic conventions}

To avoid indentation problems, simply use a good editor (Tuareg for emacs, for
example). However, here are some conventions to respect:

\begin{itemize}
\item \textbf{Indentation}: With two-spaces (don't use tabulation, set your
  editor to use spaces).
\item \textbf{Function application}: In case of multi-line call of a function,
  the arguments are indented once from the normal indentation. For example:
  \begin{OCaml}
    ...
    let r = apply
      args in
    ...
  \end{OCaml}.
\end{itemize}

\subsection*{Pattern-matching}

In the \textsf{match .. with} case, cases should be aligned with the
\textsf{match}:

\begin{OCaml}
let eval ast =
  match ast with
  | Int i -> (* code *)
  | Float f -> (* code *)
  ...
\end{OCaml}

\noindent
The first case vertical bar is mandatory for more clarity and the case statements must be aligned with the match statement.
\newline

\noindent
If a case treatment goes over a single line, it should begins under its associated pattern:

\begin{OCaml}
let eval ast =
  match ast with
  | Int i -> (* a very long
    statement *)
  | Float f -> (* another very
    long statement *)
\end{OCaml}

becomes

\begin{OCaml}
let eval ast =
  match ast with
  | Int i ->
    (* a very long
    statement *)
  | Float f ->
    (* another very
    long statement *)
\end{OCaml}

\noindent
In case \textsf{function} is used over \textsf{match .. with}, the cases must be
indented once. For example:

\begin{OCaml}
let eval = function
  | Int i -> (* code *)
...
\end{OCaml}

\noindent
And not:

\begin{OCaml}
let eval = function
           | Int i -> (* code *)
...
\end{OCaml}

\subsubsection*{Conditional Pattern-matching}

You should use the when conditional whenever you can instead of the if-then-else statement:

\begin{OCaml}
let string_of_sign value =
  match value with
  | i when i < 0 -> "-"
  | _ -> "+"
\end{OCaml}

\subsubsection*{Pattern-matching in a pattern-matching}

\medskip

Use \textsf{begin .. end} around the inner pattern-matching. The use of parenthesis should be avoided. However, it might be better to extract the inner pattern-matching inside a function.

\subsubsection*{\textsf{try .. with}}

The with is a pattern-matching for exception, it must be indented the same way.

\subsubsection*{Record pattern-matching}

Records can be matched on specific fields:
\begin{OCaml}
type t = { v : int; w : int; x : int }
...
match value with
| { v = 2; _ } -> ... 
| { w = 3; _ } -> ...
\end{OCaml}

\subsection*{The \textsf{let .. in} clause}

The \textsf{in} should be placed at the end of the expression, even in case of
long expressions. For instance:

\begin{OCaml}
...
let r = fun_call
  arg1
  arg2 in
...
\end{OCaml}

\noindent
Consecutive \textsf{let .. in} share the same indentation:

\begin{OCaml}
...
let a = ... in
let b = ... in
...
\end{OCaml}

And not:

\begin{OCaml}
...
let a = ... in
  let b = ... in
    ...
\end{OCaml}

\noindent
Actually, any expression after the \textsf{in} should be indented as its
\textsf{let}. Finally, there can't be two \textsf{let} on the same line.


\subsection*{\textsf{If .. then .. else}}

There are multiple cases. If the \textsf{if} expression can take on one line, it
should remain like this. On the other hand, the expression should have the form:

\begin{OCaml}
if cond then
  expr1
else
  expr2
\end{OCaml} 

\noindent
It is admitted that \textsf{expr1} can be on the same line as the \textsf{if} if
it fits, and respectively with \textsf{else} and \textsf{expr2} if they can fit
on one line, but it can lead to inconsistencies.

\section{Comments}

Don't comment everything, only tricky parts. Each function in the .mli must be
commented using the ocamldoc format. Here is an example from the standard
library:

\begin{OCaml}
(** [List.rev_append l1 l2] reverses [l1] and concatenates it to [l2].
    This is equivalent to {!List.rev}[ l1 @ l2], but [rev_append] is tail-recursive and more efficient. *)
val rev_append: 'a list -> 'a list -> 'a list
\end{OCaml}

Once parsed by the doc generator, the expressions between brackets are
emphasized. This kind of comments shows how to use the function, even if it is
obvious with only the type it can be really useful sometimes.

Each interface must have a header before any signature, in the typical format
(extracted from args.mli in the standard library):
\begin{OCaml}
(** Parsing of command line arguments.

   This module provides a general mechanism for extracting options and
   arguments from the command line to the program.

   ...
*)
\end{OCaml}

Explain some general behavior in it.

\section{Miscellaneous}

\begin{itemize}
\item \textbf{Global variables}: The usage of mutable global variables is
  prohibited.
\item \textbf{Duplication}: There shouldn't be two parts of code doing the same
  things, factorize.
\item \textbf{Interface}: Every .ml file must have the corresponding interface
  file. Its content must be limited to values which are used by other
  modules. 
\item \textbf{Multi-line operations}: When an expression must be declared on
  multiple lines, decompose it into \textsf{let .. in} constructions.
\item \textbf{Type annotations}: Type inference is really strong in ML, so
  annotations are not useful. They should only be used to help the compiler if
  needed, or for coercion with objects.
\item \textbf{Module opening}: Try to minimize the dependencies, if you are
  using a module in one function, use the let open.
\item \textbf{Module including}: Avoid. Prefer modules opening.
\item \textbf{Obj.magic} doesn't exists. If you think you need \textsf{Obj},
  then your solution is wrong.
\end{itemize}

\end{document}
