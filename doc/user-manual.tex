% Distributed under the MIT License.
% (See accompanying file LICENSE.txt)
% (C) Copyright NoWork team

\documentclass[12pt,a4paper]{article}

\usepackage[utf8]{inputenc}
\usepackage[american]{babel}
\usepackage[T1]{fontenc}
\usepackage{color}

\usepackage{hyperref}
\usepackage{float}

\floatstyle{ruled}
\newfloat{program}{thp}{lop}
\floatname{program}{Program}

\title{NoWork\\
User manual}
\author{NoWork development team\\[2em]}
\date\today

\begin{document}
\maketitle


\section{Introduction}
%% Pierre
NoWork is a tool that allows to define languages as a set of
rewriting and term construction rules. The main goal of NoWork
is the experimentation with the principles of programming languages
specially with respect to the rewriting rules that define them.
NoWork is in that sense similar to the Maude system. However NoWork
is the only rewriting system implementing both nominal rewriting and
non-linear pattern matching.

\subsection{Rewriting system}
%% Rémy
In computer science, we often define an operation of transition. From a
given state we go to an other state by this operation. In arithmetics,
we call this the \emph{evaluation} : we apply the calculus rules to
compute the result of a formula. In language semantics, we also call
this \emph{reduction}. A reduction step leads an expression in a given
state in a new state. Morally, in languages, the reduction operation
consists in replacing subterms of an expression by other terms.

A rewriting system is a method to describe a language by defining a
\emph{set of terms}, and the reduction operation on this language which
is a \emph{set of reduction rules}. 
To rewrite a term, you must first find the subterms which \emph{match} the left-hand side of one of the rules. Then you \emph{replace} it by the right-hand side of the choosen rules. It
could be indeterministic. In a rewriting system, it's possible to implement a lot of concepts like
$\lambda$-calculus $\beta$-Reducton, Peano's arithmetics, Milner's CCS,
De Morgan's Laws, etc. 

For instance, if we considere Peano's arithmetics, 
given the rule (\texttt{Sub(Add(X))} $\rightarrow$ \texttt{X}),
the term \texttt{Sub(Add(0))} rewrites in \texttt{0}.

\subsection{Getting started}
%% Vincent
To install this program, you will need to get \emph{Opam}, the OCaml package-manager.
It can be downloaded at \url{http://opam.ocaml.org/doc/Quick_Install.html}

Once Opam is ready to use, you have two possibilities. 

Either clone the project directly from github and install it :
\begin{verbatim}
# git clone https://github.com/ptal/nominal-workbench/
# cd nominal-workbench
# ./configure && make
\end{verbatim}

Or add the repository to your \emph{Opam} package list :
\begin{verbatim}
# opam repo add nowork git://github.com/pcouderc/nowork-repository
# opam install nowork
\end{verbatim}

There is a light emacs environment available in the \emph{data}
directory. Please refer to the emacs manual to install it in emacs.

\section{NoWork executable}
%% Vincent

The command to execute NoWork is : \texttt{nowork [options] <rules
  files>}.  A Read-Eval-Print-Loop (REPL) is launched if no argument
is given. If \texttt{.nw} files are given the program will evaluate
each file sequentially and result in a REPL that contains an initial
environment based on the definitions found in the files.

The current options are :
\begin{itemize}
\item -{}-version -- Print version and exits
\item -{}-vnum -- Print version number and exits
\item -I <dir\_path> -- Include the given directory
\item -v -- Print more informations
\item -{}-no-repl -- Doesn't start a toplevel
\item -{}-debug -- Launch nowork in debug mode, the exception backtrace is reported
\item -{}-reset-system -- Doesn't keep the environment between each evaluated files
\item -{}-no-warning -- Doesn't print the warnings.
\item -{}-help -- Display the list of options
\end{itemize}

\section{System language}
%% Matthieu
%% Presentation of the language (everything but the :command). A tutorial, example-oriented.
In this section, we will see how to describe a nominal rewriting
system for NoWork. Let us begin by showing the different options
offered by the language. Remember that the system is strongly typed.


\subsection{Kind}
Firstly, to describe our rewriting sytem, we have to describe types
manipulated by the system. Declarations of types are made using the
\verb?kind? keyword.\\
Since the system should be typed and we work on a \textbf{nominal}
rewriting system, types declared with \verb?kind? can be constructed using
\verb?type? for ``primary'' types and \verb?atom? for ``nominal''
types. We can defined parametrized type using \verb?->?. But,
constructions involving \verb?atom? and \verb?->? are (currently) forbidden.
To summarize, we have the following grammar :
\begin{verbatim}
primary-kind := type | type -> primary-kind

        kind := primary-kind | atom
\end{verbatim}

For example, some simple types :
\begin{verbatim}
kind Integer : type
kind List : type -> type
kind Couple : type -> type -> type
kind Var : atom
\end{verbatim}


\subsection{Constant}
Now that we can define types, we have to define the terms of a given type.
To define a term, we should first define the constants of the rewriting
system.\\
Defining a constant can be done by using the \verb?constant? keyword followed
by the name and the type. Be careful, the constant
type is the type of the constant for the rewriting system so it must
be defined using \verb?kind? declarations. For example :

\begin{verbatim}
kind Integer : type

constant Zero : Integer
\end{verbatim}

To use a parameterized type, we have to decide the abstraction it can be
either a generic type or a \verb?kind?.\\
To introduce genericity, we should use the \verb?forall?
construction. Otherwise, to instantiate a type we should put the type
between angle brackets. For instance :

\begin{verbatim}
kind Integer : type
kind Bool : type
kind List : type -> type
kind Couple : type -> type -> type

#generics
constant Nil : forall(A).List<A>
constant NilAssoc : forall(A,B).List<Couple<A,B>>

#instatiated
constant IntNil : List<Integer>
\end{verbatim}

\subsection{Operator}
In the same way that we did for constants, we define operators with the
\verb?operator? keyword. Operators take terms as arguments and terms
are constructed by composition between operators and constants :

\begin{verbatim}
kind Integer : type
kind Bool : type
kind List : type -> type


constant True : Bool
constant False : Bool
constant Zero : Integer
constant Nil : forall(A).List<A>

operator Cons : forall(A).A * List<A> -> List<A>
operator Hd : forall(A).List<A> -> A
operator Tl : forall(A).List<A> -> List<A>

# A term can be :
# Hd(Tl(Cons(False, Cons(True, Nil))))

# but not :
# Cons(False, Cons(Zero, Nil))
# because False and Zero does not have the same type
\end{verbatim}

\section{Rule}
Now, we want to declare rules used by the system to reduce a term.
The manner the system apply rules will be explained later using the
strategies.\\
As you would expect, we declare rule using the \verb?rule? keyword
followed by the rule's name and the term to rewrite and the term
rewrited. Such as types, we can declare generic terms in rules. This
genericity is made by using placeholder terms \verb!x?!. For example :
\begin{verbatim}
kind Integer : type
kind Bool : type
kind List : type -> type


constant True : Bool
constant False : Bool
constant Zero : Integer
constant Nil : forall(A).List<A>

operator Cons : forall(A).A * List<A> -> List<A>
operator Hd : forall(A).List<A> -> A
operator Tl : forall(A).List<A> -> List<A>

rule [hd]:
  Hd(Cons(?x, ?y)) => ?x

rule [tl]:
  Tl(Cons(?x, ?y)) => ?y
\end{verbatim}


Thus, we have all the basis language needed to describe a nominal
rewriting system. With this we can design a system
corresponding to the Lambda Calculus or the Calculus of Communicating
Systems (CCS), for example.

\section{Strategy language}
%% Roven

A basic evaluation like top-down or bottom-up will suffice for rather simple
languages like peano numbers but nothing more complicated can be defined.
Indeed for languages with avanced semantics such as lambda-calculus or CCS, we 
will need a fine grained system that control precisely the way the system 
executes the rules. For this task, we have defined a whole set of strategy 
combinators usable with a strategy language.

\subsection{A strategy}
A strategy is essentially a function that takes a set of terms, optionally other
strategies and will output a set of rewritten terms. In case of faillure, the 
resulting set will be empty and the program will signal it by an error.

Nowork contains a set of base strategies that can be combined to form more
complicated user defined stategies.

\subsection{Syntax}
The strategy language is a set of binary operators \verb|*> ;|, base strategies 
\verb|ident(args)|, user defined strategies \verb|Ident(args)| and rule 
application \verb|[rule]| combined into strategy definitions.\\

A user strategy can be defined in a nowork file with the following syntax~:
\begin{verbatim}
strategy <definition_name> : 
  <strategy combinations>
\end{verbatim}

Some examples : 

\begin{verbatim}
# parameter less strategy
strategy Name : 
  base() ; other() +> base()

# strategy with two parameters
strategy Two(x, y) :
  some_base(x) ; (Name +> y)

# recursive strategy call
strategy Loop(x) :
  x +> Loop(x)

# infinite rule application
strategy Infinite :
  Name ; Loop([rule])
\end{verbatim}

\subsection{Base strategies}
The following table summarize the base strategies available in nowork and their
associated semantic :\\

\begin{center}
    \begin{tabular}{ | l | p{6cm} | p{4cm} |}
    \hline
    Strategy & Semantic on every term in input & When it fails \\ \hline
    id() & return the term unchanged & never \\ \hline
    fail() & & always \\ \hline
    test(s) & return the term unchanged & when s fails \\ \hline
    not(s) & return the term unchanged & when s succeeds \\ \hline
    all(s) & applies s to all subterms, does nothing on constant & when one 
    application fails \\ \hline
    some(s) & try to apply s on the most subterms & when none succeed or on constant 
    \\ \hline
    one(s) & try to apply s on one subterm & when none succeed or on constant 
    \\ \hline
    s1 ; s2 & applies s1 then s2 using the result of s1 & when s1 or s2 fails 
    \\ \hline
    s1 +> s2 & applies s1 or s2 if s1 fails & when s1 and s2 fail \\ \hline
    s1 + s2 & applies s1 and s2 in parallel on the current term 
      (resulting in a set of new terms) & when s1 and s2 fail \\ \hline 
    proj(n, s) & applies s on the nth subterm &
      when this application fails \\ \hline
    [r] & applies the rule r on the term & when the rule application fails \\ \hline
    x (a variable) & applies the strategy s contained in x & when s fails \\ \hline
    \end{tabular}
\end{center}
\subsection{Examples}
In this section we will explore some examples of useful strategies defined with our
language.

\subsubsection*{Top-down and bottom-up}
We can defined those simple strategies in term of our base strategies.

\begin{verbatim}
strategy Topdown(s) :
  s ; all(Topdown(s))

strategy Bottomup(s) :
  all(Bottomup(s)) ; s
\end{verbatim}

\subsubsection*{Try}
A stategy that try to apply a strategy on a set of term and return the set unchanged
in case of faillure.

\begin{verbatim}
strategy Try(s) :
  s +> id() 
\end{verbatim}


\subsubsection*{Any}
A strategy that test if a strategy \verb|s| is applicable on any of the subterms,
returning the set of term unchanged in case of success or failling otherwise.

\begin{verbatim}
strategy Any(s) :
  test(s) +> one(Any(s))
\end{verbatim}

We can then for instance test if a rule if applicable on any subterms with
\verb|Any([some_rule])|.

\subsubsection*{While}
As a more advanced strategy, we can define a while construct, executing some strategy \verb|s| while a strategy \verb|p| succeed.

\begin{verbatim}
strategy While(p, s):
  (test(p) ; s ; While(p, s)) +> id()
\end{verbatim}


\section{Interactive language}
%% Vincent

We have at our disposition a toplevel interface that allows the user
to directly manipulate the system by introducing new rules, strategies
or terms. With this, the user is capable of interacting with the
system in a step-by-step manner. This can be use for some quick-hacks
or tests through a system that we would want to test or quickly-adding
some features to it.

This Read-Eval-Print-Loop (REPL) also contains a set of directives
that are useful for testing or debugging. A directive starts with a
colon (:). The list can be quickly found by typing the \emph{:help}
directive into the REPL.

We will proceed to the exhaustive description of every directive
present in NoWork.

\begin{itemize}
  \item :help -- Displays the help
  \item :exit -- Exits the REPL
  \item :dot <term> <"filename"> -- Outputs the hash-consed version of
    a term into a dot file graph
  \item :type <term> -- Displays the type of the term
\end{itemize}

There are also three more directives that requires some detailing.

\subsection{Testing directives}

First, we have the \emph{:test} directive. The syntax is : \emph{:test
  <term> <predicate> <result>}. This directive checks that the term
satisfies the predicate with the given result. The term denomination
is the same as the one given in the language description.

The predicates are : 
\begin{itemize}
\item \texttt{-{}-equals} -- Tests that the given term equals the resulting term.
\item \texttt{-{}-failwith} -- Checks that the given term fails with the
  given exception. They can be found in the \emph{data/error}
  directory of the project. More information about this to be found in
  the developper manual.
\end{itemize}

The second directive is \emph{:load-test <"filename"> [-{}-failwith
  <result>]}. It works very closely to the \emph{:test} directive but
takes a file instead of a term.

The last one checks that the given term matches either the given type
or the pattern structure. The syntax is : \emph{:match <term> -{}-with
  <pattern>} or \emph{:match <term> -{}-with-type <type>}.

Examples : 
\begin{verbatim}
:match App(Lambda(x, Var(x)), Lambda(y, Var(y))) 
       --with App(?T, ?T) 

:match Successor(Zero) --with-type Int
\end{verbatim}


\section{Example}
Using all the constructs described above, we are able to fully encode simple 
lambda-calculus with call-by-value semantic and CCS without nu binders\footnote{nu 
binders are excluded for simplicity purpose but could be encoded without any 
complication.}.

\subsection*{Lambda Calculus}
{\fontsize{8}{8}
\begin{verbatim}
kind Term : type
kind Variable : atom

operator Var : Variable -> Term
operator App : Term * Term -> Term
operator Lambda : [Variable]. Term -> Term
operator Subst : [Variable]. Term * Term -> Term

rule [beta] :
  App(Lambda(?x, ?T), ?U)  => Subst(?x, ?T, ?U)

rule [subst-var] :
  Subst(?x, ?T, Var(?x)) => ?T

rule [subst-nvar] :
  Subst(?x, ?T, Var(?y)) => Var(?y)

rule [app] :
  Subst(?x, ?T, App(?M, ?N)) => App(Subst(?x, ?T, ?M), Subst(?x, ?T, ?N))

rule [lambda] :
  Subst(?x, ?T, Lambda(?y, ?U)) => Lambda(?y, Subst(?x, ?T, ?U))


# extra rules for strategies
# essentially identity rules for existence test

rule [is-subst] :
  Subst(?U, ?V, ?W) => Subst(?U, ?V, ?W)

rule [is-lambda] :
  Lambda(?x, ?T) => Lambda(?x, ?T)

rule [is-left-app] : 
  App(App(?x, ?y), ?U) => App(App(?x, ?y), ?U)

rule [is-right-lambda] :
  App(?x, Lambda(?y, ?U)) => App(?x, Lambda(?y, ?U))

rule [is-app-lambda-lambda] :
  App(Lambda(?x, ?Y), Lambda(?z, ?U)) => App(Lambda(?x, ?Y), Lambda(?z, ?U))

rule [is-app] :
  App(?U, ?V) => App(?U, ?V)

# strategies

strategy IsValue :
  test([is-lambda])

strategy IsRightValue :
  test([is-right-lambda])

strategy IsLeftApp :
  test([is-left-app])

strategy IsApp :
  test([is-app])

strategy IsAppLambdaValue :
  test([is-app-lambda-lambda])

strategy Left(s) : proj(0, s)
strategy Right(s) : proj(1, s)

strategy Any(s) : 
  s +> one(Any(s))

strategy While(p, s) :
  (test(p) ; s ; While(p, s)) +> id()

strategy TrySubst :
  [subst-var] +> [subst-nvar] +> [lambda] +> id()

strategy SubstAll :
  While(Any([is-subst]), Bottomup(TrySubst))

# the main strategy : 
# - remove all pending substitutions
# - if it's an app ready for beta reduce, apply beta reduction then loop
# - if it's an app, reduce the right and left then loop
# - if it's a value do nothing

strategy ReduceAll :
    SubstAll ;
    (IsAppLambdaValue ; [beta]; ReduceAll) +>
    (IsApp ; Left(ReduceAll) ; Right(ReduceAll) ; ReduceAll) +>
    (IsValue ; id())
    
:test rewrite 
  App(
    App(Lambda(x, Var(x)), Lambda(u, Var(u))), 
    App(Lambda(z, Var(z)), Lambda(y, Var(y))))
  with Dummy
  --equal Lambda(y, Var(y))
\end{verbatim}
}
\subsection*{CCS}
{\fontsize{8}{8}
\begin{verbatim}
kind Action : type
kind ActionName : atom

kind Process : type

constant Zero : Process
constant Tau : Action

operator In : ActionName -> Action
operator Out : ActionName -> Action

operator Prefix : Action * Process -> Process
operator Sum : Process * Process -> Process
operator Par : Process * Process -> Process

rule [choice-left] :
  Sum(?X, ?Y) => ?X

rule [choice-right] :
  Sum(?X, ?Y) => ?Y

rule [par-left] :
  Par(Prefix(?X, ?Y), ?Z) => Prefix(?X, Par(?Y, ?Z))

rule [par-right] :
  Par(?X, Prefix(?Y, ?Z)) => Prefix(?Y, Par(?X, ?Z))

rule [par-zero-right] :
  Par(?X, Zero) => ?X

rule [par-zero-left] :
  Par(Zero, ?X) => ?X

# matching rules for strategies

rule [match-par] :
  Par(?X, ?Y) => Par(?X, ?Y)

rule [match-sum] :
  Sum(?X, ?Y) => Sum(?X, ?Y)

# strategies

strategy Any(s) : 
  s +> one(Any(s))

strategy While(p, s) :
  (test(p) ; s ; While(p, s)) +> id()

strategy Choice :
  [choice-left] + [choice-right]

strategy Par : 
  [par-left] + [par-right]

# main strategy:
# while there is still Par or Sum do
#   apply bottomup rules : 
#      choice (right + left), par (right + left), par zero (right + left)

strategy BottomupRules :
  While(Any([match-par] +> [match-sum]),
    Bottomup(
      Choice +>
      Par +>
      [par-zero-right] +>
      [par-zero-left] +>
      id()
    )
  )
  
:test rewrite Par(Prefix(In(c), Prefix(In(a), Zero)), Prefix(In(b), Zero)) 
  with BottomupRules 
  --equal 
    Prefix(In(c), Prefix(In(a), Prefix(In(b), Zero))) ;
    Prefix(In(c), Prefix(In(b), Prefix(In(a), Zero))) ;
    Prefix(In(b), Prefix(In(c), Prefix(In(a), Zero)))
          
\end{verbatim}

\end{document}
