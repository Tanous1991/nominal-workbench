% Distributed under the MIT License.
% (See accompanying file LICENSE.txt)
% (C) Copyright NoWork team

\documentclass[12pt,a4paper]{article}

\usepackage[utf8]{inputenc}
\usepackage[american]{babel}
\usepackage[T1]{fontenc}
\usepackage{color}

\title{NoWork\\
User manual}
\author{NoWork development team\\[2em]}
\date\today

\begin{document}
\maketitle


\section{Introduction}
%% Pierre
What is NoWork and why is it interesting with regards to other rewriting system?

\subsection{Rewriting system}
%% Rémy
In computer science, we often define an operation of transition. From a
given state we go to an other state by this operation. In arithmetics,
we call this the \emph{evaluation} : we apply the calculus rules to
compute the result of a formula. In language semantics, we also call
this \emph{reduction}. A reduction step leads an expression in a given
state in a new state. Morally, in languages, the reduction operation
consists in replacing subterms of an expression by other terms.

A rewriting system is a method to describe a language by defining a
\emph{set of terms}, and the reduction operation on this language which
is a \emph{set of reduction rules}. 
To rewrite a term, you must first find the subterms which \emph{match} the left-hand side of one of the rules. Then you \emph{replace} it by the right-hand side of the choosen rules. It
could be indeterministic. In a rewriting system, it's possible to implement a lot of concepts like
$\lambda$-calculus $\beta$-Reducton, Peano's arithmetics, Milner's CCS,
De Morgan's Laws, etc. 

For instance, if we considere Peano's arithmetics, 
given the rule (\texttt{Sub(Add(X))} $\rightarrow$ \texttt{X}),
the term \texttt{Sub(Add(0))} rewrites in \texttt{0}.

\subsection{Getting started}
%% Vincent
How do we install it (with opam or from source), which program do I need to launch and where is it?

\section{\texttt{nowork --help}}
%% Vincent
Presentation of the nowork executable and the options.

\section{System language}
%% Matthieu
%% Presentation of the language (everything but the :command). A tutorial, example-oriented.
In this section, we will see how to describe a nominal rewriting
system for NoWork. Let us begin by showing the different option
offered by the language. Remember that the system is typed.


\subsection{Kind}
Firstly, to describe our rewriting sytem, we have to describe types
manipulated by the system. Declarations of types are made using the
\verb?kind? keyword.\\
Since the system should be typed and we work on a \textbf{nominal}
rewriting system, types declared with \verb?kind? can be constructed using
\verb?type? for ``primary'' types and \verb?atom? for ``nominal''
types. We can defined parametrized type using \verb?->?. But,
constructions involving \verb?atom? and \verb?->? are (currently) forbidden.
To summarize, we have the folowing grammar :
\begin{verbatim}
primary-kind := type | type -> primary-kind

        kind := primary-kind | atom
\end{verbatim}

For example, some simple types :
\begin{verbatim}
kind Integer : type
kind List : type -> type
kind Couple : type -> type -> type
kind Var : atom
\end{verbatim}


\subsection{Constant}
Now, we can define types then we have to define term of a given type.
To define a term, we should first define constant of the rewriting
system.\\
Defining a constant is made by using the \verb?constant? keyword followed
by the constant name then the constant type. Be careful, the constant
type is the type of the constant for the rewriting system then it must
be defined using \verb?kind? declarations. For example :

\begin{verbatim}
kind Integer : type

constant Zero : Integer
\end{verbatim}

To use parametrized type, we should decide by what is parametrized the
given type : it can be parametrized by a \emph{generic} or by a
\verb?kind?.\\
To introduce a generic type, we should use the \verb?forall?
construction. Otherwise, to instantiate a type we should put the type
between \verb?<? and \verb?>?. For example :

\begin{verbatim}
kind Integer : type
kind Bool : type
kind List : type -> type
kind Couple : type -> type -> type

#generics
constant Nil : forall(A).List<A>
constant NilAssoc : forall(A,B).List<Couple<A,B>>

#instatiated
constant IntNil : List<Integer>
\end{verbatim}

\subsection{Operator}
In the same way that for constants we define operators with the
\verb?operator? keyword. Operators take terms as arguments and terms
are are constructed by composing operators and constants :

\begin{verbatim}
kind Integer : type
kind Bool : type
kind List : type -> type


constant True : Bool
constant False : Bool
constant Zero : Integer
constant Nil : forall(A).List<A>

operator Cons : forall(A).A * List<A> -> List<A>
operator Hd : forall(A).List<A> -> A
operator Tl : forall(A).List<A> -> List<A>

# A term can be :
# Hd(Tl(Cons(False, Cons(True, Nil))))

# but not :
# Cons(False, Cons(Zero, Nil))))
# because False and Zero does not have the same type
\end{verbatim}

\section{Rule}
Now, we want to declare rule used by the system to reduce a term.
How the system apply rules will be explained later using the
strategies.\\
How you expect, we declare rule using the \verb?rule? keyword
following by the rule's name then the term to rewrite and the term
rewrited. Like as type, we can declared generic term in rules. This
genericity is made by using placeholder terms \verb!x?!. For example :
\begin{verbatim}
kind Integer : type
kind Bool : type
kind List : type -> type


constant True : Bool
constant False : Bool
constant Zero : Integer
constant Nil : forall(A).List<A>

operator Cons : forall(A).A * List<A> -> List<A>
operator Hd : forall(A).List<A> -> A
operator Tl : forall(A).List<A> -> List<A>

rule [hd]:
  Hd(Cons(?x, ?y)) => ?x

rule [tl]:
  Tl(Cons(?x, ?y)) => ?y
\end{verbatim}


Thus, we have all the basis language needed to describe a nominal
rewriting system. With this we can design the rewriting system
corresponding to the lambda calculus or the calculus of communicating systems.

\subsection{Semantics}
%% Rémy

\section{Strategy language}
%% Roven

\section{Interactive language}
%% Vincent
Presentation of the interactive language (:command).

\section{Example}
%% Roven
Present full program.

\subsection{lambda Calculus}

\subsection{CCS}


\end{document}
